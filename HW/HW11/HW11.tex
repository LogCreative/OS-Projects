\documentclass[12pt,a4paper]{article}
\usepackage[UTF8]{ctex}     %先引入ctex
\usepackage[utf8]{inputenc} %再引入inputenc
\usepackage{graphicx}
\usepackage{lazylatex}
\usepackage{amsmath}
\usepackage{bookmark}
\usepackage{enumerate}
\usepackage{array}
\usepackage{xfp}
\usepackage{colortbl}
\usepackage{tikz}
\usepackage{pgfplots}

\tcbuselibrary{documentation}
\graphicspath{{img/}}
% 边距
\geometry{left=2.0cm,right=2.0cm,top=2.0cm,bottom=3.0cm}
% 大题
\newenvironment{problems}{\begin{list}{}{\renewcommand{\makelabel}[1]{\textbf{##1}\hfil}}}{\end{list}}
% 小题
\newenvironment{steps}{\begin{list}{}{\renewcommand{\makelabel}[1]{##1.\hfil}}}{\end{list}}
% 答
\providecommand{\ans}{\textbf{答}:~}
% 解
\providecommand{\sol}{\textbf{解}.~}

\newcommand{\allpart}{2325}
\providecommand{\blk}[2]{\framebox[\fpeval{round(#2/\allpart*\textwidth*0.8,2)}pt]{#1}}

\begin{document}
\title{\normalsize \underline{操作系统(D)}\\\LARGE第 14 次作业}
\author{Log Creative }
\date{\today}
\maketitle

\begin{problems}
    % \item[11.13] Suppose that a disk drive has 5,000 cylinders, numbered 0 to 4,999. The drive is currently serving a request at cylinder 2,150, and the previous request was at cylinder 1,805. The queue of pending requests, in FIFO order, is:
    
    % 2,069; 1,212; 2,296; 2,800; 544; 1,618; 356; 1,523; 4,965; 3,681
    
    % Starting from the current head position, what is the total distance (in cylinders) that the disk arm moves to satisfy all the pending requests for each of the following disk-scheduling algorithms?

    % \begin{enumerate}[a.]
    %     \item FCFS
    %     \item SCAN
    %     \item C-SCAN
    % \end{enumerate}

    % \sol \begin{enumerate}[a.]
    %     \item FCFS: 13011.
        
    %     \begin{tikzpicture}
    %         \begin{axis}[]
    %          \addplot+ [] coordinates {(2150,0)
    %         (2069,1)
    %         (1212,2)
    %         (2296,3)
    %         (2800,4)
    %         (544,5)
    %         (1618,6)
    %         (356,7) 
    %         (1523,8) 
    %         (4965,9)
    %         (3681,10)};
    %          \addplot+ [] coordinates {(1805,-1)
    %         (2150,0)};
    %         \end{axis}
    %     \end{tikzpicture}      
        
    %     \item SCAN: 7492.
        
    %     \begin{tikzpicture}
    %         \begin{axis}[xmax={4999},
    %         xmin={0},
    %         ]
    %          \addplot+ [] coordinates {(2150,0)
    %         (2296,1)
    %         (2800,2)
    %         (3681,3)
    %         (4965,4)
    %         (4999,5)
    %         (2069,6)
    %         (1618,7)
    %         (1523,8) 
    %         (1212,9)
    %         (544,10)
    %         (356,11)};
    %          \addplot+ [] coordinates {(1805,-1)
    %         (2150,0)};
    %         \end{axis}
    %         \end{tikzpicture}

    %     \item C-SCAN: 9917.
        
    %     \begin{tikzpicture}
    %         \begin{axis}[xmax={4999},
    %         xmin={0},
    %         ]
    %          \addplot+ [] coordinates {(2150,0)
    %         (2296,1)
    %         (2800,2)
    %         (3681,3)
    %         (4965,4)
    %         (4999,5)
    %         (0,6)
    %         (356,7)
    %         (544,8)
    %         (1212,9) 
    %         (1523,10) 
    %         (1618,11)
    %         (2069,12)};
    %          \addplot+ [] coordinates {(1805,-1)
    %         (2150,0)};
    %         \end{axis}
    %         \end{tikzpicture}
            
    % \end{enumerate}

    % \item[11.20] Consider a RAID level 5 organization comprising five disks, with the parity for sets of four blocks on four disks stored on the fifth disk. Howmany blocks are accessed in order to perform the following?
    
    % \begin{enumerate}[a.]
    %     \item A write of one block of data 
    %     \item A write of seven continuous blocks of data
    % \end{enumerate}
    
    % \sol \begin{enumerate}[a.]
    %     \item 1+4=5.
    %     \item 7+4$\times$7=35.
    % \end{enumerate}

    % \item[12.5] How does DMA increase system concurrency? How does it complicate hardware design?
    
    % \ans 直接内存访问(DMA)启用时,CPU 直接将 DMA 命令块的地址写到 DMA 控制器,然后可以继续其他工作,这样 DMA 就可以直接在没有主 CPU 的帮助的情况下执行传输,从而提高系统的一致性。硬件需要专门为 DMA 设置总线接口,周期窃取也复杂了硬件设计。

    % \item[13.7] Consider a system that supports 5,000 users. Suppose that you want toallow 4,990 of these users to be able to access one file. 
    % \begin{enumerate}[a.]
    %     \item How would you specify this protection scheme in UNIX?
    %     \item Can you suggest another protection scheme that can be used more effectively for this purpose than the scheme provided by UNIX?
    % \end{enumerate} 

    % \ans \begin{enumerate}[a.]
    %     \item 创建 4990 条 ACL 或者将 4990 用户放在同一个组中,对这个组进行访问设置。
    %     \item 默认都可以访问,但是添加10个用户的“黑名单”。
    % \end{enumerate}

    \item[14.1] Consider a file currently consisting of 100 blocks. Assume that the file-control block (and the index block, in the case of indexed allocation) is already in memory. Calculate how many disk I/O operations are required for contiguous, linked, and indexed (single-level) allocation strategies, if, for one block, the following conditions hold. In the contiguous-allocation case, assume that there is no room to grow at the beginning but there is room to grow at the end. Also assume that the block information to be added is stored in memory.
    \begin{enumerate}[a.]
        \item The block is added at the beginning.
        \item The block is added in the middle.
        \item The block is added at the end.
        \item The block is removed from the beginning.
        \item The block is removed from the middle.
        \item The block is removed from the end.
    \end{enumerate} 

    \sol 连续需要移动。链接需要释放该块。索引只需要修改索引,添加时写入,删除时不管。

    \begin{tabular}{c|ccc}
            & Contiguous & Linked & Indexed \\
            \hline
        a.  & 201        & 1      & 1       \\
        b.  & 101        & 52(查找需要50)     & 1       \\
        c.  & 1          & 3(写入、上一尾+这一尾) & 1\\
        d.  & 198        & 1      & 0       \\
        e.  & 98         & 52     & 0       \\
        f.  & 0(只改长度) & 100(寻找需要99+更改) & 0 \\
    \end{tabular}
\end{problems}

\end{document}
