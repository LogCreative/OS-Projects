\documentclass[12pt,a4paper]{article}
\usepackage[UTF8]{ctex}     %先引入ctex
\usepackage[utf8]{inputenc} %再引入inputenc
\usepackage{graphicx}
\usepackage{lazylatex}
\usepackage{amsmath}
\usepackage{bookmark}
\usepackage{enumerate}
\usepackage{array}
\usepackage{xfp}
\usepackage{colortbl}

\tcbuselibrary{documentation}
\graphicspath{{img/}}
% 边距
\geometry{left=2.0cm,right=2.0cm,top=2.0cm,bottom=3.0cm}
% 大题
\newenvironment{problems}{\begin{list}{}{\renewcommand{\makelabel}[1]{\textbf{##1}\hfil}}}{\end{list}}
% 小题
\newenvironment{steps}{\begin{list}{}{\renewcommand{\makelabel}[1]{##1.\hfil}}}{\end{list}}
% 答
\providecommand{\ans}{\textbf{答}:~}
% 解
\providecommand{\sol}{\textbf{解}.~}

\newcommand{\allpart}{20}
\providecommand{\blk}[2]{\framebox[\fpeval{round(#2/\allpart*\textwidth*0.8,2)}pt]{$P_#1$}}

\begin{document}
\title{\normalsize \underline{操作系统(D)}\\\LARGE第 5 次作业}
\author{李子龙 518070910095}
\date{\today}
\maketitle

这次作业做的比较差。要分清抢占式和非抢占式。

% \begin{problems}
%     \item[5.4] Consider the following set of processes, with the length of the CPU burst
%     time given in milliseconds:

%     \begin{tabular}{ccc}
%         \underline{Process} & \underline{Burst Time} & \underline{Priority} \\
%         $P_1$ & 2 & 2 \\
%         $P_2$ & 1 & 1 \\
%         $P_3$ & 8 & 4 \\
%         $P_4$ & 4 & 2 \\
%         $P_5$ & 5 & 3
%     \end{tabular}

%     The processes are assumed to have arrived in the order $P_1$, $P_2$, $P_3$, $P_4$, $P_5$,
%     all at time 0.

%     \begin{steps}
%         \item[a] Draw four Gantt charts that illustrate the execution of these processes
%         using the following scheduling algorithms: FCFS, SJF, nonpreemptive
%         priority (a larger priority number implies a higher
%         priority), and RR (quantum = 2).
%         \item[b] What is the turnaround time of each process for each of the
%         scheduling algorithms in part a?
%         \item[c] What is the waiting time of each process for each of these scheduling
%         algorithms?
%         \item[d] Which of the algorithms results in the minimum average waiting
%         time (over all processes)?
%     \end{steps}

%     \sol
%         \begin{steps}
%             \item[a] 甘特图如下:
             
%             \renewcommand{\allpart}{20}
%             \begin{tabular}{cc}
%                 FCFS & \blk{1}{2}\blk{2}{1}\blk{3}{8}\blk{4}{4}\blk{5}{5}\\
%                 SJF & \blk{2}{1}\blk{1}{2}\blk{4}{4}\blk{5}{5}\blk{3}{8}\\
%                 Prority & \blk{3}{8}\blk{5}{5}\blk{1}{2}\blk{4}{4}\blk{2}{1}\\
%                 RR & \blk{1}{2}\blk{2}{1}\blk{3}{2}\blk{4}{2}\blk{5}{2}\blk{3}{2}\blk{4}{2}\blk{5}{2}\blk{3}{2}\blk{5}{1}\blk{3}{2}
%             \end{tabular} 

%             \item[b] 周转时间表:
%             \begin{center}
%             \begin{tabular}{c|rrrr}
%                 & FCFS & SJF & Priority & RR \\
%                 \hline
%                 $P_1$ & 2 & 3 & 15& 2\\
%                 $P_2$ & 3 & 1 & 20& 3\\
%                 $P_3$ & 11& 7 & 8 & 20\\
%                 $P_4$ & 15& 12& 19& 13\\
%                 $P_5$ & 20& 20& 13& 18
%             \end{tabular}
%             \end{center}

%             \item[c] 等待时间表:
%             \begin{center}
%             \begin{tabular}{c|rrrr}
%                 & FCFS & SJF & Priority & RR \\
%                 \hline
%                 $P_1$ & 0 & 1 & 13&0\\
%                 $P_2$ & 2 & 0 & 19&2\\
%                 $P_3$ & 3 & 12& 0 &12\\
%                 $P_4$ & 11& 3 & 15&9\\
%                 $P_5$ & 15& 7 & 8 &13
%             \end{tabular}
%             \end{center}
%             \item[d] 等待平均时间:
%             \begin{center}
%             \begin{tabular}{c|rrrr}
%                 & FCFS & SJF & Priority & RR \\
%                 \hline
%                 Waiting Avg & 6.2& 4.6& 11& 7.2\\ 
%             \end{tabular}
%             \end{center} 

%             所以 SJF 拥有更短的平均等待时间。
%         \end{steps}

%     \item[5.5] The following processes are being scheduled using a preemptive, roundrobin
%     scheduling algorithm.

%     \begin{tabular}{cccc}
%         \underline{Process} & \underline{Priority} & \underline{Burst} & \underline{Arrival} \\
%         $P_1$ & 40& 20& 0\\
%         $P_2$ & 30& 25& 25\\
%         $P_3$ & 30& 25& 30
% \\        $P_4$ & 35& 15& 60
% \\        $P_5$ & 5 &10 &100
% \\        $P_6$ & 10& 10& 105
%     \end{tabular}

%     Each process is assigned a numerical priority,with a higher number indicating
% a higher relative priority. In addition to the processes listed below,
% the system also has an \textbf{idle task} (which consumes no CPU resources and
% is identified as $P_{idle}$). This task has priority 0 and is scheduled whenever
% the system has no other available processes to run. The length of a time quantum is 10 units. If a process is preempted by a higher-priority
% process, the preempted process is placed at the end of the queue.

% \begin{steps}
%     \item[a] Show the scheduling order of the processes using a Gantt chart.
%     \item[b] What is the turnaround time for each process?
%     \item[c]  What is the waiting time for each process?
%     \item[d] What is the CPU utilization rate?
% \end{steps}

% \sol \begin{steps}
%     \item[a] 甘特图如下:
    
%     \renewcommand{\allpart}{125}
%     \begin{tabular}{cc}
%         RR & \blk{1}{10}\blk{1}{10}\blk{{idle}}{10}\blk{2}{10}\blk{3}{10}\blk{2}{10}\blk{4}{10}\blk{3}{5}\blk{2}{5}\blk{4}{5}\blk{{idle}}{10}\blk{{idle}}{10}\blk{6}{10}\blk{5}{10}
%     \end{tabular}

%     \item[b] 周转时间表:
    
%     \begin{tabular}{c|rrrrrr}
%         & $P_1$ & $P_2$ & $P_3$ & $P_4$ & $P_5$ & $P_6$ \\
%         \hline
%         finish & 20 & 60 & 75 & 85 & 125 & 115\\
%         \rowcolor{green!20} turnaround & 20 & 35 & 45 & 25 & 25 & 10
%     \end{tabular}

%     \item[c] 等待时间:
     
%     \begin{tabular}{c|rrrrrr}
%          & $P_1$ & $P_2$ & $P_3$ & $P_4$ & $P_5$ & $P_6$ \\
%         \hline
%         arrival & 0 & 25 & 30 & 60 & 100 & 105 \\
%         \rowcolor{green!20} waiting & 0 & 30 & 20 & 10 & 15 & 0 
%     \end{tabular}

%     \item[d] CPU 利用率:
%     \begin{equation*}
%         \left(1 - \frac{30}{125} \right)\times 100 \%=76\% 
%     \end{equation*} 

% \end{steps}

% \item[5.10] The traditional UNIX scheduler enforces an inverse relationship between
% priority numbers and priorities: the higher the number, the lower the
% priority. The scheduler recalculates process priorities once per second
% using the following function:
% \begin{quotation}
%     Priority = (recent CPU usage / 2) + base
% \end{quotation}
% where base = 60 and \emph{recent CPU usage} refers to a value indicating how
% often a process has used the CPU since priorities were last recalculated.
% Assume that recent CPU usage for process $P_1$ is 40, for process $P_2$ is 18,
% and for process $P_3$ is 10. What will be the new priorities for these three
% processes when priorities are recalculated? Based on this information,
% does the traditional UNIX scheduler raise or lower the relative priority
% of a CPU-bound process?

% \sol 

% \begin{tabular}{c|rrr}
%      & $P_1$ & $P_2$ & $P_3$ \\
%      \hline
%     recent CPU Usage & 40 & 18 & 10 \\
%     \rowcolor{green!20} priority & 80 & 69 & 65 
% \end{tabular}

% UNIX 降低了常用进程的优先级。

% \item[5.18] The following processes are being scheduled using a preemptive,
% priority-based, round-robin scheduling algorithm.

% \begin{tabular}{cccc}
%     \underline{Process} & \underline{Priority} & \underline{Burst} &  \underline{Arrival}\\
% $P_1$ & 8 & 15& 0\\
% $P_2$ & 3 & 20& 0\\
% $P_3$ & 4 & 20& 20\\
% $P_4$ & 4 & 20& 25\\
% $P_5$ & 5 & 5 &45\\
% $P_6$ & 5 & 15& 55
% \end{tabular}

% Each process is assigned a numerical priority, with a higher number indicating
% a higher relative priority. The scheduler will execute the highest priority
% process. For processes with the same priority, a round-robin
% scheduler will be used with a time quantum of 10 units. If a process is
% preempted by a higher-priority process, the preempted process is placed
% at the end of the queue.

% \begin{steps}
%     \item[a] Show the scheduling order of the processes using a Gantt chart.
%     \item[b]  What is the turnaround time for each process?
%     \item[c]  What is the waiting time for each process?
% \end{steps}

% \sol \begin{steps}
%     \item[a] 甘特图如下:
    
%     \renewcommand{\allpart}{95}
%     \blk{1}{10}\blk{2}{10}\blk{1}{5}\blk{3}{10}\blk{4}{10}\blk{5}{5}\blk{2}{10}\blk{6}{10}\blk{3}{10}\blk{4}{10}\blk{6}{5}
    
%     \item[b] 周转时间表:
     
%     \begin{tabular}{c|rrrrrr}
%         & $P_1$ & $P_2$ & $P_3$ & $P_4$ & $P_5$ & $P_6$ \\
%         \hline
%         finish & 25 & 60 & 80 & 90 & 50 & 95\\
%         \rowcolor{green!20} turnaround & 25 & 60 & 60 & 65 & 5 & 40
%     \end{tabular}

%     \item[c] 等待时间表: 
    
%     \begin{tabular}{c|rrrrrr}
%         & $P_1$ & $P_2$ & $P_3$ & $P_4$ & $P_5$ & $P_6$ \\
%        \hline
%        arrival & 0 & 0 & 20 & 25 & 45 & 55 \\
%        \rowcolor{green!20} waiting & 10 & 40 & 40 & 45 & 0 & 25
%    \end{tabular}
     
% \end{steps}

% \item[5.20] Which of the following scheduling algorithms could result in starvation?
% \begin{steps}
%     \item[a] First-come, first-served
%     \item[b]  Shortest job first
%     \item[c]  Round robin
%     \item[d]  Priority
% \end{steps} 

% \sol 最短作业优先调度(b.) 和 优先级调度(d.)。因为前者可以作业时间很长的任务可能会永远得不到执行,后者优先级很低的任务可能永远得不到执行。

% \end{problems}

\end{document}
