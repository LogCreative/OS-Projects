\documentclass[12pt,a4paper]{article}
\usepackage[UTF8]{ctex}     %先引入ctex
\usepackage[utf8]{inputenc} %再引入inputenc
\usepackage{graphicx}
\usepackage{lazylatex}
\usepackage{amsmath}
\usepackage{bookmark}
\usepackage{enumerate}
\usepackage{array}
\usepackage{xfp}
\usepackage{colortbl}
\usepackage{tikz}

\tcbuselibrary{documentation}
\graphicspath{{img/}}
% 边距
\geometry{left=2.0cm,right=2.0cm,top=2.0cm,bottom=3.0cm}
% 大题
\newenvironment{problems}{\begin{list}{}{\renewcommand{\makelabel}[1]{\textbf{##1}\hfil}}}{\end{list}}
% 小题
\newenvironment{steps}{\begin{list}{}{\renewcommand{\makelabel}[1]{##1.\hfil}}}{\end{list}}
% 答
\providecommand{\ans}{\textbf{答}:~}
% 解
\providecommand{\sol}{\textbf{解}.~}

\newcommand{\allpart}{2325}
\providecommand{\blk}[2]{\framebox[\fpeval{round(#2/\allpart*\textwidth*0.8,2)}pt]{#1}}

\begin{document}
\title{\normalsize \underline{操作系统(D)}\\\LARGE第 10 次作业}
\author{李子龙 518070910095}
\date{\today}
\maketitle

\begin{problems}
    \item[10.5] Consider the page table for a system with 12-bit virtual and physical addresses and 256-byte pages.
    
    \begin{tabular}{cc}
        \hline
        Page & Page Frame \\
        \hline
        0 & - \\
        1 & 2 \\
        2 & C \\
        3 & A \\
        4 & - \\
        5 & 4 \\
        6 & 3 \\
        7 & - \\
        8 & B \\
        9 & 0 \\
        \hline
    \end{tabular}

    The list of free page frames is $D,E,F$(that is,$D$ is at the head of the list,$E$ is second, and $F$ is last). A dash for a page frame indicates that thepage is not in memory.
    
    Convert the following virtual addresses to their equivalent physicaladdresses in hexadecimal. All numbers are given in hexadecimal.

    \begin{itemize}
        \item 9EF
        \item 111
        \item 700
        \item 0FF
    \end{itemize}
    \item[10.7] Consider the two-dimensional array \verb"A":
    \begin{verbatim}
        int A[][] = new int[100][100];
    \end{verbatim} 
    where \verb"A[0][0]" is at location 200 in a paged memory system with pagesof size 200. A small process that manipulates the matrix resides in page0 (locations 0 to 199). Thus, every instruction fetch will be from page 0.
    
    For three page frames, how many page faults are generated by the following array-initialization loops? Use LRU replacement, and assume
    that page frame 1 contains the process and the other two are initially empty.
    \begin{enumerate}[a.]
        \item \begin{verbatim}
            for (int j = 0; j < 100; j++)
                for (int i = 0; i < 100; i++)
                    A[i][j] = 0;
        \end{verbatim}
        \item \begin{verbatim}
            for (int i = 0; i < 100; i++)
                for (int j = 0; j < 100; j++)
                    A[i][j] = 0;
        \end{verbatim}
    \end{enumerate}
    \item[10.8] Consider the following page reference string:
    \begin{center}
        1, 2, 3, 4, 2, 1, 5, 6, 2, 1, 2, 3, 7, 6, 3, 2, 1, 2, 3, 6
    \end{center}
    How many page faults would occur for the following replacementalgorithms, assuming one, two, three, four, five, six, and seven frames?Remember that all frames are initially empty, so your first unique pageswill cost one fault each.
    \begin{itemize}
        \item LRU replacement
        \item FIF Oreplacement
        \item Optimal replacement
    \end{itemize}
    \item[10.9]  Consider the following page reference string:
    \begin{center}
        7, 2, 3, 1, 2, 5, 3, 4, 6, 7, 7, 1, 0, 5, 4, 6, 2, 3, 0 , 1
    \end{center} 
    Assuming demand paging with three frames, how many page faultswould occur for the following replacement algorithms?
    \begin{itemize}
        \item LRU replacement
        \item FIF Oreplacement
        \item Optimal replacement
    \end{itemize}
\end{problems}

\end{document}