%example.tex

\documentclass[12pt,a4paper]{article}
\usepackage[UTF8]{ctex}     %先引入ctex
\usepackage[utf8]{inputenc} %再引入inputenc
% \usepackage{CJKutf8}
\usepackage{graphicx}
\usepackage{lazylatex}
\usepackage{lipsum}
\usepackage{amsmath}
\usepackage{bookmark}
\usepackage{listings}
\tcbuselibrary{documentation}
\graphicspath{{img/}}
\lstset{
    language=c,numbers=left, numberstyle=\tiny, keywordstyle=\color{blue!70}, commentstyle=\color{red!50!green!50!blue!50}, rulesepcolor=\color{red!20!green!20!blue!20},basicstyle=\ttfamily,breaklines=true,
    columns=flexible
}
\begin{document}
% \begin{CJK}{UTF}{song}
\title{第 1 周作业}
\author{LogCreative}
\date{\today}

\maketitle
% \tableofcontents

\section{公共库: 多项式类 \texttt{Polynomial}}

任何一个实函数$F$都可以通过插值多项式的方法得到一个拟合的还不错的多项式,所以在这里为了简化我们的问题,就直接使用多项式函数求解问题。

当然,如果需要使用真正的超越方程,可以使用类似于下面的代码创建内联函数,提高调用效率:
\begin{code}{c}
    inline double f(double x){ return x*exp(x)-1; }
\end{code}

对于一个多项式,用户依次输入幂次、各项系数,就可以构建一个 \texttt{Polynomial} 多项式类,并由此来创建新的头文件库\lib{polynomial.h}。

\inputminted[]{cpp}{code/polynomial.h}

\section{问题 1: Romberg 积分法}

对于给定的函数$f$与区间$[a,b]$,为了计算积分$\int_a^b f(x)\text{d}x$,根据 \href{https://en.wikipedia.org/wiki/Romberg\%27s_method}{Wikipedia 中关于 Romberg 积分法的描述},可以得到关于该积分法的以下递推公式:

$$h_n=\frac{1}{2^n}(b-a)$$

\begin{align*}
    R(0,0) &= h_1 (f(a)+f(b)) \\ 
    R(n,0) &= \frac{1}{2}R(n-1,0) + h_n\sum_{k=1}^{2^{n-1}}f(a+(2k-1)h_n)\\
    R(n,m) &=\frac{1}{4^m -1}\left( R(n,m-1) - R(n-1,m-1) \right) \\ 
    &\text{where } n\geq m \text{ and } m\geq 1
\end{align*}


定义了一个 \texttt{Romberg} 类,使用动态规划的方式存储之前的数据,最后输出 $R(n,n)$ 作为结果。

用户输入两个数据,程序会使用 Romberg 积分法输出。

由于计算多项式的值速度会比较慢,所以增加一个进度显示。

\inputminted[]{cpp}{code/Romberg.cpp}
% \lstinputlisting[language=c]{}

\begin{literal}
Please input the power of polynomial:2
Please input the coefficients of polynomial(start from the constant):-1 2 5
Please input the fineness of integral:1000
Please input the start and end of intergral interval:1 5
Calculating Process: 100%
226.667
\end{literal}

\section{问题2: 求解函数零点}

\subsection{二分法}

函数使用公共的头文件输入。

定义了一个函数 \texttt{FzeroSolver} 用于寻找函数零点,为了简化问题,我们只对输入区间两端函数值为异号的情况进行计算,否则会抛出异常 \texttt{SameSymbolException} 。(\texttt{Matlab} 的 \texttt{fzero} 函数也是这么做的。)

当超过迭代次数的时候,将会抛出异常 \texttt{OutOfIterationLimitException},并显示超出迭代次数。

\subsection{牛顿法}

也使用了牛顿法对零点进行计算,使用端点值作为迭代初值用于计算。

% \lstinputlisting[language=c]{code/fzero.cpp}
\inputminted[]{cpp}{code/fzero.cpp}

\begin{literal}
Please input the power of polynomial:2
Please input the coefficients of polynomial(start from the constant):-11 1 5
Please input the solving interval:1 5
The zero of function(Binary Method):1.38661
The zero of function(Newton's Method):1.38661
\end{literal}

\begin{literal}
Please input the power of polynomial:2
Please input the coefficients of polynomial(start from the constant):1 1 5
Please input the solving interval:1 5
The zero of function(Binary Method):The symbol of the function values of each side should be different.
nan
The zero of function(Newton's Method):Out of iteration limit.
nan
\end{literal}

% \end{CJK}
\end{document}
