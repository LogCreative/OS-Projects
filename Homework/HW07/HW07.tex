\documentclass[12pt,a4paper]{article}
\usepackage[UTF8]{ctex}     %先引入ctex
\usepackage[utf8]{inputenc} %再引入inputenc
\usepackage{graphicx}
\usepackage{lazylatex}
\usepackage{amsmath}
\usepackage{bookmark}
\usepackage{enumerate}
\usepackage{array}
\usepackage{xfp}
\usepackage{colortbl}

\tcbuselibrary{documentation}
\graphicspath{{img/}}
% 边距
\geometry{left=2.0cm,right=2.0cm,top=2.0cm,bottom=3.0cm}
% 大题
\newenvironment{problems}{\begin{list}{}{\renewcommand{\makelabel}[1]{\textbf{##1}\hfil}}}{\end{list}}
% 小题
\newenvironment{steps}{\begin{list}{}{\renewcommand{\makelabel}[1]{##1.\hfil}}}{\end{list}}
% 答
\providecommand{\ans}{\textbf{答}:~}
% 解
\providecommand{\sol}{\textbf{解}.~}

\newcommand{\allpart}{20}
\providecommand{\blk}[2]{\framebox[\fpeval{round(#2/\allpart*\textwidth*0.8,2)}pt]{$P_#1$}}

\begin{document}
\title{\normalsize \underline{操作系统(D)}\\\LARGE第 7 次作业}
\author{李子龙 518070910095}
\date{\today}
\maketitle

\begin{problems}
    \item[7.8] The Linux kernel has a policy that a process cannot hold a spinlock while attempting to acquire a semaphore. Explain why this policy is in place.
    
    \ans 一个进程拥有旋转锁时不能获取信号量,是因为旋转锁处于忙等待状态的同时,获取信号量也有可能会导致忙等待,该进程所使用的CPU已经处于忙状态时不应该再叠加一次忙状态,可能会显著增加等待时间,甚至宕机。
    \item[7.11] Discuss the tradeoff between fairness and throughput of operations in the readers–writers problem. Propose a method for solving the readers–writers problem without causing starvation.
    
    \sol 如果为了保持公平,读者和作者一次只能执行一个,但这会导致吞吐量的显著降低。而如果设置读者优先级高于作者,就可以保持可以多个进程可以同时读取,作者与读者互斥。

    无饥饿的解法:可以设置两个信号量
    \begin{code}{c++}
        semaphore writer = 1;
        semaphore reader = 1;
    \end{code}
    作者进程如下:
    \begin{code}{c++}
    do{
        wait(reader);
        wait(writer);

        /* writing is performed */

        signal(writer);
        signal(reader);
    } while (true);
    \end{code}
    读者进程如下:
    \begin{code}{c++}
    do{
        wait(reader);
        wait(mutex);
        read_count++;
        if(read_count==1)
            wait(writer);
        signal(mutex);
        signal(reader);

        /* reading is performed */

        wait(mutex);
        read_count--;
        if(read_count==0)
            signal(writer);
        signal(mutex);
    } while (true);
    \end{code}
    
    当一个作者在做写操作时,第一个读者会在 \verb"writer" 上等待,其余的读者会在\verb"reader" 上等待。作者完成写操作时,读者都可以进行读操作,\verb"reader"被解开,\verb"writer"被锁上。只有当所有的读操作都结束后,\verb"writer"会被解开。

    这种方式不会发生死锁。如果作者在\verb"reader"或者\verb"writer"上等待,就一定会有读者在临界区,而且这些读者同时在临界区,并足够将 \verb"read_count" 重置为 0,解开\verb"writer"以及通过 \verb"signal(reader)" 解开 \verb"reader"。如果读者在\verb"reader"上等待,就一定有作者在临界区,而且此时 \verb"writer" 一定处于锁上状态,否则 \verb"reader" 将会被解开,那么作者一定会完成,以解开 \verb"writer" 和 \verb"reader"。

    这种解法作者不会饿死。这个时候作者将会共同占用 \verb"reader" 导致其不会被无限阻塞,但仍然满足作者和读者互斥。一旦 \verb"reader"被解开,作者就有可能准备进入临界区解锁,其余读者的数量将会有限,一旦最后一个读者完成读操作,\verb"writer" 就会被解开,作者的等待将是有限的。
    \item[7.16] The C program \verb"stack-ptr.c" (available in the source-code download) contains an implementation of a stack using a linked list. An example of its use is as follows:
    \begin{code}{c}
        StackNode *top = NULL;
        push(5, &top);
        push(10, &top);
        push(15, &top);
        
        int value = pop(&top);
        value = pop(&top);
        value = pop(&top);
    \end{code} 
    This program currently has a race condition and is not appropriate for a concurrent environment. Using Pthreads mutex locks (described in Section 7.3.1), fix the race condition.

    \sol \begin{code}{c}
        /* Before all processes start. */
        #include<pthread.h>

        pthread_mutex_t mutex;
        pthread_mutex_init(&mutex,NULL);
    \end{code}
    \begin{code}{c}
        /* For every process. */
        int main(){
            pthread_mutex_lock(&mutex);
            StackNode *top = NULL;
            push(5, &top);
            push(10, &top);
            push(15, &top);
            
            int value = pop(&top);
            value = pop(&top);
            value = pop(&top);
            pthread_mutex_unlock(&mutex);
        }
    \end{code} 
\end{problems}

\end{document}